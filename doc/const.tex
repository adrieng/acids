\section{Const typing}

The goal of const typing is to enable some form of compile-time evaluation while
avoiding the Heptagon quagmire where we basically have two subtly different
languages that do not mix well. There are only two ground const types
distinguishing between expressions that can be evaluated to a compile-time
constant, and those who cannot.

\subsection{Expressions}

It is a very simple type system with the following grammar:
\grammar{
  \grammardef{\tau}{\alpha}{const type variables}
  \grammarcase{\St}{stationary stream}
  \grammarcase{\Dt}{non-stationary stream}
}

The type system involves simple sub-typing because a compile-time constant can
be used anywhere a non-constant expression was expected: $\St \mathrel{<:} \Dt$.

\subsection{Nodes}

The programmer can declare const nodes, that is nodes that will not get clocked
and will be carried around in intermediate files. Non-const nodes are black
boxes and therefore cannot be used to define compile-time constants.

A non-const node may not have const inputs. Conversely, its const outputs are
converted to dynamic since they may not be used to define compile-time constants.

The following program is thus rejected:
\begin{verbatim}
let node f () = false

let node g () = 0 when <(b)> where rec b = f ()
\end{verbatim}
Whereas the following one is accepted:
\begin{verbatim}
let const node f () = false

let node g () = 0 when <(b)> where rec b = f ()
\end{verbatim}
