\chapter{Code generation and optimization}

\section{Overview}

The middle-end of Acid Synchrone uses an intermediate language called \nir{}. It
is a first-order data-flow language similar to Acid Synchrone, with the
following differences:
\begin{itemize}
\item it is in three-address forms: subexpressions are only variables;
\item there are no tuples and function calls are n-ary;
\item it has a notion of scheduling-constrained equation lists called \textit{blocks}.
\end{itemize}

\paragraph{Block}

A block is a list of equations that cannot be scheduled independently. TODO

\section{Translation process}

The goal of this part of the compiler is to produce valid imperative code. It
has to deal with the following questions:
\begin{itemize}
\item How to remove integer clocks?
\item How to sequentialize mutually recursive equations?
\end{itemize}

The middle-end applies the following mandatory transformations to \nir{}, in this
order:
\begin{enumerate}
\item slice the code so that each node has an unique base clock;
\item put each equation in its own block;
\item materialize clock variables as integer variables;
\item put buffers on variable uses in nested blocks;
\item schedule blocks hierarchically.
\end{enumerate}

\subsection{Clock Slicing}

A node $f$ clock-polymorphic in two variables $\alpha$ and $\alpha'$ gets sliced
into two nodes $f_\alpha$ and $f_\alpha'$. Equations are easy to slice since
there are no tuples anymore, except for node calls. Calls to $f$ in some node
$g$ get sliced into two calls. Example, applied to Acid Synchrone rather than
NIR:
\begin{verbatim}
f :: ('a * 'a1) -> ('a * 'a1)

let node g x = f (x, x)
g :: 'a -> ('a * 'a)

let node h x = f (x, 2)
h :: 'a -> ('a * 'a1)
\end{verbatim}
gets translated to:
\begin{verbatim}
let node g_a x = (y, z) where
  rec y = f_a x
  and z = f_a1 x

let node h_a x = y where
  rec y = f_a0 x

let node h_a1 () = y where
  rec y = f_a1 2
\end{verbatim}

\subsection{Block formation}

The goal of block formation is to put blocks around some equations. Depending on
the equation's operation, it may:
\begin{itemize}
\item do nothing (var, buffer)
\item WIP
\end{itemize}

\section{Optimizations}

I want to implement the following optimizations:
\begin{itemize}
\item Copy elimination
\item Constant folding
\item Data-flow minimization
\item Buffer sharing
\item Block transformations
  \begin{itemize}
  \item Fusion
  \item Fission
  \end{itemize}
\end{itemize}